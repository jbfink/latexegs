% for live demo
\documentclass{article}
\title{My first document}
%\author{John Fink \\ email \href{mailto:jfink@mcmaster.ca}{jfink@mcmaster.ca}}
\author{John Fink}
\date{July 19th, 2022}

% packages and package setup go here
\usepackage{tikz}
\usepackage{chemfig}
\usepackage[version=4]{mhchem}
\usepackage{siunitx} % required for decimal point alignment
\usepackage{mathtools} % for eliminating automatic numbering of math

%\usepackage{hyperref} % if you want your href email thing to work in the commented out author line.
\sisetup{
	round-mode = places, % Rounds numbers
	round-precision = 2, % to 2 places
}


\begin{document}

\maketitle
	This is a document!
	
% Footnote example
	This is a sentence with a footnote!\footnote{A footnote}
	
	Footnotes are automagically numbered!\footnote{Wow!}\\

% Diagrams with tikz
% examples from https://www.overleaf.com/learn/latex/LaTeX_Graphics_using_TikZ%3A_A_Tutorial_for_Beginners_(Part_1)%E2%80%94Basic_Drawing 
\section{Diagrams with tikz}
Diagrams!!\\
Start with \verb|\begin{tikzpicture}..\end{tikzpicture}|! Draw a line! Feel the rush!

\begin{tikzpicture}
	\draw (0,0) -- (4,0);
\end{tikzpicture}

Draw a box! The excitement, will it ever stop?

\begin{tikzpicture}
	\draw (0,0) -- (4,0) -- (4,4) -- (0,4) -- (0,0);
\end{tikzpicture}

No!!! We must \textbf{keep going}!! Here's the same box but ... better?

\begin{tikzpicture}
	\draw (0,0) rectangle (4,4);
\end{tikzpicture}

A circle??? Centred?

\begin{center}
	\begin{tikzpicture}
		\draw (2,2) circle (3cm) node {circle?};
	\end{tikzpicture}
\end{center}

% Tables
% from https://latex-tutorial.com/tutorials/tables/
% Rows can be added by adding a line after "3", e.g. like "4 & 3432.1234 & d \\"
% Columns can be added by adding to the Value line like "& \textbf{Value 4}\\" and then specifying the label "& $\delta$ \\, and then adding more values to the rows "& d" etc.
\section{tables}
\begin{table}[h!] % h! eliminates float
\begin{center}
	\caption{Your first table.}
	\label{tab:table1}
	\begin{tabular}{l|c|r} % <-- Alignments: 1st column left, 2nd middle and 3rd right, with vertical lines in between
		\textbf{Value 1} & \textbf{Value 2} & \textbf{Value 3}\\
		$\alpha$ & $\beta$ & $\gamma$ \\
		\hline
		1 & 1110.1 & a\\
		2 & 10.1 & b\\
		3 & 23.113231 & c\\
	\end{tabular}
\end{center}
\end{table}

Now let's \textbf{align those decimals} with our \textit{siunitx} package.

\begin{table}[h!]
	\begin{center}
		\caption{Your second table.}
		\label{tab:table2}
		\begin{tabular}{l|S|r} % <-- Note: middle value changed to S for centering
			\textbf{Value 1} & \textbf{Value 2} & \textbf{Value 3}\\
			$\alpha$ & $\beta$ & $\gamma$ \\
			\hline
			1 & 1110.1 & a\\
			2 & 10.1 & b\\
			3 & 23.113231 & c\\
		\end{tabular}
	\end{center}
\end{table}




% Math
% See https://en.m.wikibooks.org/wiki/LaTeX/Mathematics
\section{math}
Remember, math can be done \textit{inline}! $f(x)=(x+a)(x+b)$\footnote{inline!!!} or set by itself using e.g. \verb|\begin{equation}...\end{equation}|:

\begin{equation*} % The * suppresses the automatic numbering - mathtools provides this.
	 \frac{1}{\sqrt{x}}
\end{equation*}

More math! I don't know what I'm doing! $k_{n+1} = n^2 + k_n^2 - k_{n-1}$




% Chemistry
% Chemfig and mhchem examples: https://en.m.wikibooks.org/wiki/LaTeX/Chemical_Graphics
\section{chemistry}
Chemistry!


\chemfig{A-B}

\chemfig{A<B}

\chemfig{C(-[:0]H)(-[:90]H)(-[:180]H)(-[:270]H)}

\ce{H2O}

\ce{NO3-}

\ce{(NH4)2S)}



\end{document}